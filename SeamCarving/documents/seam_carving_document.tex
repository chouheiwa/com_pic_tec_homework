%! Author = chouheiwa
%! Date = 2022/10/31

% Preamble
\documentclass[UTF8]{article} %article 文档
\usepackage{ctex}  %使用宏包(为了能够显示汉字)
\usepackage{hyperref}
\usepackage{graphicx}
\usepackage{geometry}
\geometry{a4paper,scale=0.8}
\usepackage{listings}
\usepackage{color}
\definecolor{dkgreen}{rgb}{0,0.6,0}
\definecolor{gray}{rgb}{0.5,0.5,0.5}
\definecolor{mauve}{rgb}{0.58,0,0.82}
\lstset{frame=tb,
    language=Python,
    aboveskip=3mm,
    belowskip=3mm,
    showstringspaces=false,
    columns=flexible,
    basicstyle={\small\ttfamily},
    numbers=left,%设置行号位置none不显示行号
%numberstyle=\tiny\courier, %设置行号大小
    numberstyle=\tiny\color{gray},
    keywordstyle=\color{blue},
    commentstyle=\color{dkgreen},
    stringstyle=\color{mauve},
    breaklines=true,
    breakatwhitespace=true,
    escapeinside=``,%逃逸字符(1左面的键),用于显示中文例如在代码中`中文...`
    tabsize=4,
    extendedchars=false %解决代码跨页时,章节标题,页眉等汉字不显示的问题
}

% Packages
\usepackage{amsmath}

% Document
\title{Seam Carving算法实践}
\author{chouheiwa}
\date{2022/10/31}
\linespread{1.5}

% Document
\begin{document}
    \maketitle
    \tableofcontents

    \section{算法简介}
    \subsection{删除算法}
    Seam Carving算法不同于传统的图像分割算法,其主要是基于图像内容感知的图像感知算法。其主要思想是通过动态规划的方式,找到图像中的能量最小的路径,然后将其删除,从而达到图像缩放的目的。其算法的主要步骤如下:
    \begin{enumerate}
        \item 计算图像的能量矩阵 \label{em:1}
        \item 通过动态规划的方式,找到能量最小的连续路径 \label{em:2}
        \item 从原图中删除对应的路径
        \item 重复上述步骤,直到达到目标图像的大小
    \end{enumerate}

    上述步骤中的~\ref{em:1}可以使用多种方式来计算图像的能量矩阵,本文中使用了3种计算能量计算能量的方法,包括:
    \begin{enumerate}
        \item 简单算法(个人理解,即将当前像素的上下左右像素的差值取平均数作为当前像素的能量)
        \item Sobel算子(其主要原理是使用了图像梯度)
        \item Forward Energy算法~\cite{rubinstein2008improved}
    \end{enumerate}




    \bibliography{main}
    \bibliographystyle{plain}

\end{document}